\documentclass[11pt, dvipdfmx]{jsbook}
%jsbookクラスを利用。
%本文は11ptとし、dviファイルをpdfファイルに変換するためにdvipdfmxを利用することを宣言している。
%美文書33-35p.

\usepackage{09_others/packages}
%様々なパッケージを読み込むという指令を書き込んだパッケージを読み込んでいる。
%詳細は当該ファイル参照。

\usepackage{09_others/ngskshien_style}
%自前の命令やjsbookクラスへの修正などが書き込まれたパッケージを読み込んでいる。
%詳細は当該ファイル参照。

\begin{document}
\setcounter{tocdepth}{2}
%subsectionまで目次に表示する設定にしている。

\frontmatter
%ここから\mainmatterが来るまで前付け扱いになり、ページの表記がローマ数字になる。本修論に前書きは存在しないので、目次のみページ数表記がローマ数字になる。「はじめに」などを作る場合この部分にchapterを作る。

\tableofcontents
%ここに目次を入れる。

\mainmatter
%ここから本文(ページ数表記がアラビア数字になる)。

\chapter{序論}\label{cp:introduction}
序論ではページの体裁を紹介する。詳しいTipsは\ref{cp:techniques}章を参照のこと。
\section{最初に}\label{sec:first}
あああああああああ。
\subsection{これがサブセクション}\label{sub:unique}
これがサブセクションの例です。面倒だからsubsectionは一つしか書いてないし貴重な例です。見出し分けをどれくらい細かい階層にするかは個人の趣味や指導教員の趣味(研究室の伝統)によると思います。細かい分類が必要ない場合は単に使わなければよいですし、これ以上細かい分類が必要な場合はスタイルファイルを参考に深い階層を定義してください。
\section{次に}
こんにちは\cite{LaTeXbibunsho}。←これはこのTeXサンプルを作るのに一番参考にした参考書だよ。英文の論文も引用してみるよ\cite{mayor1995jupiter}。←ちなみにこれは2019年ノーベル物理学賞のやつだよ(読んでない)。

改ページしたときの挙動を見るためにここから「あ」を連呼します。あああああああああああああああああああああああああああああああああああああああああああああああああああああああああああああああああああああああああああああああああああああああああああああああああああああああああああああああああああああああああああああああああああああああああああああああああああああああああああああああああああああああああああああああああああああああああああああああああああああああああああああああああああああああああああああああああああああああああああああああああああああああああああああああああああああああああああああああああああああああああああああああああああああああああああああああああああああああああああああああああああああああああああああああああああああああああああああああああああああああああああああああああああああああああああああああああああああああああああああああああああああああああああああああああああああああああああああああああああああああああああああああああああああああああああああああああああああああああああああああああああああああああああああああああああああああああああああああああああああああああああああああああああああああああああああああああああああああああああああああああああああああああああああああああああああああああああああああああああああああああああああああああああああああああああああああああああああああああああああああああああああああああああああああああああああああああああああああああああああああああああああああああああああああああああああああああああああああああああああああああああああああああああああああああああああああああああああああああああああああああああああああああああああああああああああああああああああああああああああああああああああああああああああああああああああああああああああああああああああああああああああああああああああああああああああああああああああああああああああああああああああああああああああああああああああああああああああああああああああああああああああああああああああああああああああああああああああああああああああああああああああああああああああああああああああああああああああああああああああああああああああああああああああああああああああああああああああああああああああああああああああああああああああああああああああああああああああああああああああああああああああああああああああああああああああああああああああああああああああああああああああああああああああああああああああああああああああああああああああああああああああああああああああああああああああああああああああああああああああああああああああああああああああああああああああああああああああああああああああああああああああああああああああああああああああああああああああああああああああああああああああああああああああああああああああああああああああああああああああああああああああああああああああああああああああああああああああああああああああああああああああああああああああああああああああああああああああああああああああああああああああああああああああああああああああああああああああああああああああああああああああああああああゐああああああああああああああああああああああああああああああああああああああああああああああああああああああああああああああああああああああああああああああああああああああああああああああああああああああああああああああああああああああああああああああああああああああああああああああああああああああああああああああああああああああああああああああああああああああああああああああああああああああああああああああああああああああああああああああああああああああああああああああああああああああああああああぬあああああああああああああああああああああああああああああああああああああああああああああああああああああああああああああああああああああああああああああああああああああああああああああああああああああああああああああああああああああああああああああああああああああああああああああああああああああああああああああああああああああああああああああああああああああああああああああああああああああああああああああああああああああああああああああああああああああああああああああああああああああああああああああああああああああああああああああああああああああああああああああああああああああああああああああああああああああああああああああああああああああああああああああああああああああああああああああああああああああああああああああああああああああああああああああああああああああああああああああああああああああああああああああああああああああああああああああああああああああああああああああああああああああああああああああああああああああああああああああああああめああああああああああああああああああああああああああああああああああああああああああああああああああああああああああああああああああああああああああああああああああああああああああああああああああああああああああああああああああああああああああああああああああああ
\chapter{テクニック集}\label{cp:techniques}
\section{図表}
図は読み込むコマンドを用意しています。そうすると入力が楽であるばかりか、形式を一定にすることができます。別に私と同じようにする必要はありませんが、一つの方策として提案しておきます。ちなみに図\ref{pic:fig1}が単純に図を1つ載せるコマンドで、図\ref{pic:fig2}および図\ref{pic:fig3}は横に2つ図を並べるコマンドです。figure環境やminipage環境をもっと複雑に制御することも可能なので、やってみたい人はやってみてください。コマンドの詳細はスタイルファイルを見てください。
\onefigure{0.3}{fig1.jpg}{これが図の例です。図に意味はありません。}{pic:fig1}
%onefigureは、図の幅(columwidth単位)、図の名前(パス)、キャプション、引用ラベルの順に引数を取るコマンドです。
\twofigure{0.49}{fig2.jpg}{これも図の例です。}{pic:fig2}
{0.49}{fig3.jpg}{これがこの文章最後の図です。}{pic:fig3}
%twofigureは、onefigureを二回繰り返すような引数の順になっています。
\section{数式}
数式に関しては\cite{LaTeXbibunsho}に詳しいですし、インターネットにいくらでも情報がありますし、わりと分野による部分もありますから、いくらか情報を紹介するにとどめておきます。具体的に何か入力したいものがある場合はとにかくググればなんとかなります。

執筆の効率化のため、良く出てくるような数式で入力が面倒なものはコマンド化しておくのがよいでしょう。私がコマンド化したものを特定されない程度に書いておくと次のようなものがあります。1階の常微分と偏微分は毎回入力するのが面倒なので、式\ref{eq:derivative}のように2つ引数を取るコマンドにしてしまいました~\footnote{ソースコード見ないとわからないけど}。
\begin{align}\label{eq:derivative}
\OD{x}{t},~\PD{\psi}{t}
\end{align}
他には2次の正方行列や2次の縦ベクトルも式\ref{eq:matrix}のようにコマンド化してしまいました。
\begin{align}\label{eq:matrix}
\twomat
{1}{0}
{0}{1}
\twovec
{a}
{b}
=
\twovec
{a}
{b}
\end{align}%行列は↑のように可読性を高める工夫をしたりします。行数が増えてかえって見にくいこともあります。
以上のように、出現頻度の高い数式はコマンド化すると便利です。ただし、ソースコードの可読性が低くなったりうっかり同じコマンド名を使ってしまったりすると場合もあるので注意が必要です。
\section{単位}
\section{化学式}
物性物理や化学の分野だと化学式や構造式を扱うことも多いと思います。私自身それほど詳しいわけではないので、ほぼ次のブログ記事の引き写しであることをことわっておきます(https://doratex.hatenablog.jp/)。まず、mhchemというパッケージを読み込むことで化学式の扱いが非常に楽になります(パッケージ無しでもmathrmと下付き上付きを繰り返すだけなので難しくはないですが)。たとえば受験化学に出てきそうな化学反応を示してみると式\ref{eq:chem1}のようになります。
\begin{align}\label{eq:chem1}
\ce{CH3COOH <=> CH3COO- + H+}
\end{align}%詳しい使い方はhttps://doratex.hatenablog.jp/entry/20131203/1386068127 をご覧になってください。

次に示すように、chemfigというパッケージを利用することで有機化合物の構造式を示すことも一応できます。私は使ったことがありませんが、有機半導体とかcharge-transferとかやってる人は使うのかもしれません。一応せっかくなので、式\ref{eq:chem2}に有名な有機半導体の誘導体(?)であるTTF(TetraThiaFulvalene)の構造式を示しておきます。わりと面倒なコマンド入力が必要ですが、\LaTeX 内で完結するし数式環境でもそれ以外でも使えるので反応式にも入れられるし、まあ便利っちゃ便利です。GUIで構造式を入力したらchemfig形式で返してくれるソフトとかないですかね…………。反応機構とか書き始めるとbstファイル並みの闇が待っているのでやりたい人は頑張ってください。
\begin{align}\label{eq:chem2}
\chemfig{
*5(-S-(=*5(-S-=-S-))-S-=)
}
\end{align}%詳しいことはhttps://doratex.hatenablog.jp/entry/20141212/1418393703 を見てください(投げっぱなし)。
\section{相互参照}
\section{引用}
%\chapter{三論}
\section{三初に}
あああああああああ。いいうううう。
\section{次の次の次の次に}
HELLO!
%ここで本文のファイルを読み込んでいる。
%各行の先頭に'%'を付けたり付けなかったりすることで特定の章のみをコンパイルすることが可能(例ではchapter3をコメントアウトしている)。

\appendix
%ここから付録(appendix)であることを示し、章番号の扱いが変わる(付録Aや付録Bのような表記になる)。実験物理系であれば、詳細な実験のセットアップや統計処理プログラムの概要を示したり、本文に入れるには些末すぎるが分量がある結果などを示すのに使えるかもしれない。とはいえ修論には必要ない場合も多そう。

\chapter{おまけの文章}
\section{最後の一つ前に}
まみむめも。おはよう。
\section{これもセクション}
ああああああああああああああああああああああああああああああああああああああああああああああああああああああああ。
%付録Aを読み込む。複数読み込むとB,C,D…となる。

\backmatter
%ここから後付けの内容であることを示し、章番号が表示されなくなる。謝辞や参考文献などはここから始める。普通の修論では付けないと思うけれど索引もここ。

\bibliographystyle{junsrt}
%参考文献の表示スタイルを読み込む。このファイルを編集するのは非常に大変なので早めに取り組んでおくか、最適なものを早めに見つけておくことが必要。とりあえず無難なjunsrt.bstにしておく。たいていの場合思い通りのスタイルにはならないので自分で書き換える必要があると思います。つらい。

\bibliography{09_others/articles,09_others/books}
%参考文献を読み込む。Mendeleyなどの文献管理ソフトから吐き出したものがarticlesで、自分で入力した日本語の文献(主に教科書など)をbooksに入力する前提。

\chapter{謝辞}
謝辞にはセクションを付けてないよ。それはそれとして、このTeXソースを作るにあたって書籍やインターネットから非常に多くのことを吸収しました。この文章を書いている時点でそれをリストアップしたり包括的に紹介したりはしていませんが、とりあえず感謝感謝でございます。ありがとうね。
%謝辞を読み込む。

\end{document}