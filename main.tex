\documentclass[11pt, dvipdfmx]{jsbook}
%jsbookクラスを利用。
%本文は11ptとし、dviファイルをpdfファイルに変換するためにdvipdfmxを利用することを宣言している。
%美文書33-35p.

\usepackage{09_others/packages}
%様々なパッケージを読み込むという指令を書き込んだパッケージを読み込んでいる。
%詳細は当該ファイル参照。

\usepackage{09_others/ngskshien_style}
%自前の命令やjsbookクラスへの修正などが書き込まれたパッケージを読み込んでいる。
%詳細は当該ファイル参照。

\begin{document}
\setcounter{tocdepth}{2}
%subsectionまで目次に表示する設定にしている。

\frontmatter
%ここから\mainmatterが来るまで前付け扱いになり、ページの表記がローマ数字になる。本修論に前書きは存在しないので、目次のみページ数表記がローマ数字になる。「はじめに」などを作る場合この部分にchapterを作る。

\tableofcontents
%ここに目次を入れる。

\mainmatter
%ここから本文(ページ数表記がアラビア数字になる)。

\chapter{序論}
\section{最初に}
あああああああああ。
\subsection{これがサブセクション}
これがサブセクションの例です。面倒だからsubsectionは一つしか書いてないし貴重な例だよ。
\section{次に}
こんにちは\cite{LaTeXbibunsho}。←これはこのTeXサンプルを作るのに一番参考にした参考書だよ。英文の論文も引用してみるよ\cite{mayor1995jupiter}。←ちなみにこれは2019年ノーベル物理学賞のやつだよ(読んでない)。

改ページしたときの挙動を見るためにここから「あ」を連呼します。あああああああああああああああああああああああああああああああああああああああああああああああああああああああああああああああああああああああああああああああああああああああああああああああああああああああああああああああああああああああああああああああああああああああああああああああああああああああああああああああああああああああああああああああああああああああああああああああああああああああああああああああああああああああああああああああああああああああああああああああああああああああああああああああああああああああああああああああああああああああああああああああああああああああああああああああああああああああああああああああああああああああああああああああああああああああああああああああああああああああああああああああああああああああああああああああああああああああああああああああああああああああああああああああああああああああああああああああああああああああああああああああああああああああああああああああああああああああああああああああああああああああああああああああああああああああああああああああああああああああああああああああああああああああああああああああああああああああああああああああああああああああああああああああああああああああああああああああああああああああああああああああああああああああああああああああああああああああああああああああああああああああああああああああああああああああああああああああああああああああああああああああああああああああああああああああああああああああああああああああああああああああああああああああああああああああああああああああああああああああああああああああああああああああああああああああああああああああああああああああああああああああああああああああああああああああああああああああああああああああああああああああああああああああああああああああああああああああああああああああああああああああああああああああああああああああああああああああああああああああああああああああああああああああああああああああああああああああああああああああああああああああああああああああああああああああああああああああああああああああああああああああああああああああああああああああああああああああああああああああああああああああああああああああああああああああああああああああああああああああああああああああああああああああああああああああああああああああああああああああああああああああああああああああああああああああああああああああああああああああああああああああああああああああああああああああああああああああああああああああああああああああああああああああああああああああああああああああああああああああああああああああああああああああああああああああああああああああああああああああああああああああああああああああああああああああああああああああああああああああああああああああああああああああああああああああああああああああああああああああああああああああああああああああああああああああああああああああああああああああああああああああああああああああああああああああああああああああああああああああああああああああああああゐああああああああああああああああああああああああああああああああああああああああああああああああああああああああああああああああああああああああああああああああああああああああああああああああああああああああああああああああああああああああああああああああああああああああああああああああああああああああああああああああああああああああああああああああああああああああああああああああああああああああああああああああああああああああああああああああああああああああああああああああああああああああああああぬあああああああああああああああああああああああああああああああああああああああああああああああああああああああああああああああああああああああああああああああああああああああああああああああああああああああああああああああああああああああああああああああああああああああああああああああああああああああああああああああああああああああああああああああああああああああああああああああああああああああああああああああああああああああああああああああああああああああああああああああああああああああああああああああああああああああああああああああああああああああああああああああああああああああああああああああああああああああああああああああああああああああああああああああああああああああああああああああああああああああああああああああああああああああああああああああああああああああああああああああああああああああああああああああああああああああああああああああああああああああああああああああああああああああああああああああああああああああああああああああああめああああああああああああああああああああああああああああああああああああああああああああああああああああああああああああああああああああああああああああああああああああああああああああああああああああああああああああああああああああああああああああああああああああ
\chapter{次論}
\section{次に次に}
あああああああああ。いいいいいい。
\section{何か埋める}
おはようございます。
%\chapter{三論}
\section{三初に}
あああああああああ。いいうううう。
\section{次の次の次の次に}
HELLO!
%ここで本文のファイルを読み込んでいる。
%各行の先頭に'%'を付けたり付けなかったりすることで特定の章のみをコンパイルすることが可能(例ではchapter3をコメントアウトしている)。

\appendix
%ここから付録(appendix)であることを示し、章番号の扱いが変わる(付録Aや付録Bのような表記になる)。実験物理系であれば、詳細な実験のセットアップや統計処理プログラムの概要を示したり、本文に入れるには些末すぎるが分量がある結果などを示すのに使えるかもしれない。とはいえ修論には必要ない場合も多そう。

\chapter{おまけの文章}
\section{最後の一つ前に}
まみむめも。おはよう。
\section{これもセクション}
ああああああああああああああああああああああああああああああああああああああああああああああああああああああああ。
%付録Aを読み込む。複数読み込むとB,C,D…となる。

\backmatter
%ここから後付けの内容であることを示し、章番号が表示されなくなる。謝辞や参考文献などはここから始める。普通の修論では付けないと思うけれど索引もここ。

\bibliographystyle{junsrt}
%参考文献の表示スタイルを読み込む。このファイルを編集するのは非常に大変なので早めに取り組んでおくか、最適なものを早めに見つけておくことが必要。とりあえず無難なjunsrt.bstにしておく。たいていの場合思い通りのスタイルにはならないので自分で書き換える必要があると思います。つらい。

\bibliography{09_others/articles,09_others/books}
%参考文献を読み込む。Mendeleyなどの文献管理ソフトから吐き出したものがarticlesで、自分で入力した日本語の文献(主に教科書など)をbooksに入力する前提。

\chapter{謝辞}
謝辞にはセクションを付けてないよ。それはそれとして、このTeXソースを作るにあたって書籍やインターネットから非常に多くのことを吸収しました。この文章を書いている時点でそれをリストアップしたり包括的に紹介したりはしていませんが、とりあえず感謝感謝でございます。ありがとうね。
%謝辞を読み込む。

\end{document}